\documentclass[a4paper, 12pt]{report}

\usepackage[dvipsnames]{xcolor}

%%%%%%%%%%%%%%%%
% Set Variables %
%%%%%%%%%%%%%%%%

\def\useItalian{0}  % 1 = Italian, 0 = English

\def\courseName{Mathematical Logic for Computer Science}

\def\coursePrerequisites{TODO}

\def\book{TODO}

% \def\authorName{Simone Bianco}
% \def\email{bianco.simone@outlook.it}
% \def\github{https://github.com/Exyss/university-notes}
% \def\linkedin{https://www.linkedin.com/in/simone-bianco}

\def\authorName{Alessio Bandiera}
\def\email{alessio.bandiera02@gmail.com}
\def\github{https://github.com/aflaag-notes}
\def\linkedin{https://www.linkedin.com/in/alessio-bandiera-a53767223}

%%%%%%%%%%%%
% Packages %
%%%%%%%%%%%%

\usepackage{../../packages/Nyx/nyx-packages}
\usepackage{../../packages/Nyx/nyx-styles}
\usepackage{../../packages/Nyx/nyx-frames}
\usepackage{../../packages/Nyx/nyx-macros}
\usepackage{../../packages/Nyx/nyx-title}
\usepackage{../../packages/Nyx/nyx-intro}

%%%%%%%%%%%%%%
% Title-page %
%%%%%%%%%%%%%%

\logo{../../packages/Nyx/logo.png}

\if\useItalian1
    \institute{\curlyquotes{\hspace{0.25mm}Sapienza} Università di Roma}
    \faculty{Ingegneria dell'Informazione,\\Informatica e Statistica}
    \department{Dipartimento di Informatica}
    \ifdefined\book
        \subtitle{Appunti integrati con il libro \book}
    \fi
    \author{\textit{Autore}\\\authorName}
\else
    \institute{\curlyquotes{\hspace{0.25mm}Sapienza} University of Rome}
    \faculty{Faculty of Information Engineering,\\Informatics and Statistics}
    \department{Department of Computer Science}
    \ifdefined\book
        \subtitle{Lecture notes integrated with the book \book}
    \fi
    \author{\textit{Author}\\\authorName}
\fi


\title{\courseName}
\date{\today}

% \supervisor{Linus \textsc{Torvalds}}
% \context{Well, I was bored\ldots}

\addbibresource{./references.bib}

%%%%%%%%%%%%
% Document %
%%%%%%%%%%%%

\begin{document}
    \maketitle

    % The following style changes are valid only inside this scope 
    {
        \hypersetup{allcolors=black}
        \fancypagestyle{plain}{%
        \fancyhead{}        % clear all header fields
        \fancyfoot{}        % clear all header fields
        \fancyfoot[C]{\thepage}
        \renewcommand{\headrulewidth}{0pt}
        \renewcommand{\footrulewidth}{0pt}}

        \romantableofcontents
    }

    \introduction

    %%%%%%%%%%%%%%%%%%%%%

    \chapter{TODO}

    \section{TODO}

    placeholder \todo{missing introduction}
    
    \begin{frameddefn}{Theory}
        A \tbf{theory} is a --- possibly infinite --- set of propositions (or hypothesis).
    \end{frameddefn}

    As a natural extension of the \tit{satisfiability} property discussed above, a theory $T$ will be said to be \tbf{satisfiable} --- written as $T \in \SAT$ if and only if $$\exists \alpha \quad \alpha(T) = 1$$ which is equivalent of saying that $$\exists \alpha \quad \forall F \in T \quad \alpha (F) = 1$$ (note that the assignment $\alpha$ must be the same for all the propositions $F$ of $T$).

    Additionally, for infinite theories we can define another property.

    \begin{frameddefn}{Finite satisfiability}
        An infinite theory $T$ is said to be \tbf{finitely satisfiable} if and only if $$\forall T' \subset T \ \mathrm{finite} \quad T' \in \SAT$$ We will denote with $\FINSAT$ the set of all finitely satisfiable theories.
    \end{frameddefn}
    
    However, the following theorem will prove that \tit{satisfiability} and \tit{finite satisfiability} are actually \tbf{equivalent}.

    \begin{framedthm}{Compactness theorem}
        Given an infinite theory $T$, it holds that $$T \in \SAT \iff T \in \FINSAT$$
    \end{framedthm}

    \begin{proof}
        In this proof we will assume that the propositions of the infinite theory $T$ are \tit{countably infinite}, however in its general form this theorem can be proved even without this assumption.

        Since the direct implication of this statement is trivially true by definition, we just need to prove the converse implication.

        \claim{
            Given a theory $T \in \FINSAT$, and a proposition $A$, it must hold that $T \cup \{A\} \in \FINSAT$ or $T \cup \{\lnot A \} \in \FINSAT$.
        }{
            By way of contradiction, assume that $T \cup \{A\}, T \cup \{\lnot A\} \notin \FINSAT$.

            By definition of finite satisfiability, if $T \cup \{A\} \notin \FINSAT$, then there must exist a \tit{finite} sub-theory $T_0 \subset T \cup \{A\}$ such that $T_0 \in \UNSAT$. Note that $T \in \FINSAT$, therefore if $T \cup \{A\} \notin \FINSAT$ then it must be that $A \in T_0$. Let $\widehat{T_0}$ be the theory such that $T_0 := \widehat{T_0} \cup \{A\}$; then $$T_0 := \widehat{T_0} \cup \{A\} \in \UNSAT \iff \forall \alpha  \quad \alpha(T_0) = 0$$ which implies that $$\forall \alpha \quad \alpha(\widehat{T_0}) = 1 \implies \alpha(A) = 0$$

            Analogously, we can apply the same reasoning for $T \cup \{\lnot A\}$, and we get that there must exist a \tit{finite} sub-theory $T_1 \subset T \cup \{\lnot A\}$ such that $T_1 := \widehat{T_1} \cup \{\lnot A\} \in \UNSAT$, which implies that $$\forall \alpha \quad \alpha (\widehat{T_1}) = 1 \implies \alpha(\lnot A) = 0$$

            Lastly, since $\widehat{T_0} \cup \widehat{T_1} \subset T \in \FINSAT$, by finite satisfiability of $T$ there must exist an assignment $\alpha$ such that $\alpha(\widehat{T_0} \cup \widehat{T_1}) = 1$, and therefore $\alpha(\widehat{T_0}) = \alpha(\widehat{T_1}) = 1$. However, for previous observations this implies that $\alpha(A) = \alpha(\lnot A) = 0 \ \lightning$.
        }

        TODO todo{da finire}
    \end{proof}

    The statement of this theorem is equivalent of the following.

    \begin{framedcor}{}
        Given an infinite theory $T$, and a proposition $A$, it holds that $$T \models A \iff \exists T' \subset T \ \mathrm{finite} \quad T' \models A$$
    \end{framedcor}

    \begin{proof}
        placeholder \todo{prove their equivalence}
    \end{proof}
   
    \printbibliography % UNCOMMENT FOR BIBLIOGRAPHY

\end{document}
