\documentclass[a4paper, 12pt]{report}

\usepackage[dvipsnames]{xcolor}

%%%%%%%%%%%%%%%%
% Set Variables %
%%%%%%%%%%%%%%%%

\def\useItalian{0}  % 1 = Italian, 0 = English

\def\courseName{Mathematical Logic for Computer Science}

\def\coursePrerequisites{TODO}

\def\book{TODO}

% \def\authorName{Simone Bianco}
% \def\email{bianco.simone@outlook.it}
% \def\github{https://github.com/Exyss/university-notes}
% \def\linkedin{https://www.linkedin.com/in/simone-bianco}

\def\authorName{Alessio Bandiera}
\def\email{alessio.bandiera02@gmail.com}
\def\github{https://github.com/aflaag-notes}
\def\linkedin{https://www.linkedin.com/in/alessio-bandiera-a53767223}

%%%%%%%%%%%%
% Packages %
%%%%%%%%%%%%

\usepackage{../../packages/Nyx/nyx-packages}
\usepackage{../../packages/Nyx/nyx-styles}
\usepackage{../../packages/Nyx/nyx-frames}
\usepackage{../../packages/Nyx/nyx-macros}
\usepackage{../../packages/Nyx/nyx-title}
\usepackage{../../packages/Nyx/nyx-intro}

%%%%%%%%%%%%%%
% Title-page %
%%%%%%%%%%%%%%

\logo{../../packages/Nyx/logo.png}

\if\useItalian1
    \institute{\curlyquotes{\hspace{0.25mm}Sapienza} Università di Roma}
    \faculty{Ingegneria dell'Informazione,\\Informatica e Statistica}
    \department{Dipartimento di Informatica}
    \ifdefined\book
        \subtitle{Appunti integrati con il libro \book}
    \fi
    \author{\textit{Autore}\\\authorName}
\else
    \institute{\curlyquotes{\hspace{0.25mm}Sapienza} University of Rome}
    \faculty{Faculty of Information Engineering,\\Informatics and Statistics}
    \department{Department of Computer Science}
    \ifdefined\book
        \subtitle{Lecture notes integrated with the book \book}
    \fi
    \author{\textit{Author}\\\authorName}
\fi


\title{\courseName}
\date{\today}

% \supervisor{Linus \textsc{Torvalds}}
% \context{Well, I was bored\ldots}

\addbibresource{./references.bib}

%%%%%%%%%%%%
% Document %
%%%%%%%%%%%%

\begin{document}
    \maketitle

    % The following style changes are valid only inside this scope 
    {
        \hypersetup{allcolors=black}
        \fancypagestyle{plain}{%
        \fancyhead{}        % clear all header fields
        \fancyfoot{}        % clear all header fields
        \fancyfoot[C]{\thepage}
        \renewcommand{\headrulewidth}{0pt}
        \renewcommand{\footrulewidth}{0pt}}

        \romantableofcontents
    }

    \introduction

    %%%%%%%%%%%%%%%%%%%%%

    \chapter{TODO}

    \section{Introduction}

    placeholder \todo{missing introduction}

    \begin{frameddefn}{Assignment}
        Given a proposition $A$, an \tbf{assignment} $\alpha$ is a function that assigns either 0 or 1 to all $A$'s propositional variables.
    \end{frameddefn}

    This definition can be inductively extended to \tbf{propositions} themselves; for instance if $A$ and $B$ are two propositions, then $$\alpha(A \land B) = \soe{ll}{1 & \alpha(A) = \alpha(B) = 1 \\ 0 & \alpha(A) = 0 \lor \alpha(B) = 0}$$ and so on for all the other Boolean connectives. Two propositions $A$ and $B$ are said to be \tbf{equivalent} --- written as $A \equiv B$ --- if and only if $\forall \alpha \quad \alpha(A) = \alpha(B)$.

    \begin{frameddefn}{Satisfiability}
        A proposition $A$ is said to be \tbf{satisfiable} if there exists an assignment $\alpha$ of its propositional variables such that $\alpha(A) = 1$. If there is no assignment that satisfies $A$, $A$ is said to be \tbf{unsatisfiable}, and if $A$ is satisfied for any assignment $\alpha$, then $A$ is said to be a \tbf{tautology}.

        We will denote with \SAT, \UNSAT and \TAUT respectively the sets of all satisfiable propositions, all unsatisfiable propositions and all tautologies.
    \end{frameddefn}

    Using symbols, we have that

    \begin{itemize}
        \item $A \in \SAT \iff \exists \alpha \quad \alpha (A) = 1$
        \item $A \in \UNSAT \iff \nexists \alpha \quad \alpha (A) = 1 \iff \forall \alpha \quad \alpha (A) = 0$
        \item $A \in \TAUT \iff \forall \alpha \quad \alpha(A) = 1 \iff \nexists \alpha \quad \alpha (A) = 0$
    \end{itemize}
    
    The concept of satisfiability is strictly related to the concept of \tbf{logical consequence}, which is defined as follows.

    \begin{frameddefn}{Logical consequence}
        Given the propositions $A_1, \ldots, A_n, A$, we say that $A$ is a \tbf{logical consequence} of $A_1, \ldots, A_n$ if whenever $A_1, \ldots, A_n$ are true, $A$ is also true. We will indicate this concept as follows: $$A_1, \ldots, A_n \models A$$
    \end{frameddefn}

    From its definition, the concept of logical consequence can be alternatively be expressed in terms of \tit{unsatisfiability} and \tit{tautology}.

    \begin{framedthm}{}
        Given the formulas $A_1, \ldots, A_n, A$, the following statements are equivalent:

        \begin{itemize}
            \item $A_1, \ldots, A_n \models A$
            \item $(A_1 \land \ldots \land A_n \rightarrow A) \in \TAUT$
            \item $(A_1 \land \ldots \land A_n \land A) \in \UNSAT$
        \end{itemize}
    \end{framedthm}

    \subsection{Theories}

    \begin{frameddefn}{Theory}
        A \tbf{theory} is a --- possibly infinite --- set of propositions (or hypothesis).
    \end{frameddefn}

    As a natural extension of the \tit{satisfiability} property previously discussed, a theory $T$ will be said to be \tbf{satisfiable} --- written as $T \in \SAT$ if and only if $$\exists \alpha \quad \alpha(T) = 1$$ which is equivalent of saying that $$\exists \alpha \quad \forall F \in T \quad \alpha (F) = 1$$ note that the assignment $\alpha$ must be the same for all the propositions $F$ of $T$.

    Additionally, for infinite theories we can define another property.

    \begin{frameddefn}{Finite satisfiability}
        An infinite theory $T$ is said to be \tbf{finitely satisfiable} if and only if $$\forall T' \subset T \ \mathrm{finite} \quad T' \in \SAT$$ We will denote with $\FINSAT$ the set of all finitely satisfiable theories.
    \end{frameddefn}
    
    However, the following theorem will prove that \tit{satisfiability} and \tit{finite satisfiability} are actually \tbf{equivalent}.

    \begin{framedthm}{Compactness theorem}
        Given an infinite theory $T$, it holds that $$T \in \SAT \iff T \in \FINSAT$$
    \end{framedthm}

    \begin{proof}
        In this proof we will assume that the propositions of the infinite theory $T$ are \tit{countably infinite}, however in its general form this theorem can be proved even without this assumption.

        Since the direct implication of this statement is trivially true by definition, we just need to prove the converse implication.

        \claim{
            Given a theory $T \in \FINSAT$, and a proposition $A$, it must hold that $T \cup \{A\} \in \FINSAT$ or $T \cup \{\lnot A \} \in \FINSAT$.
        }{
            By way of contradiction, assume that $T \cup \{A\}, T \cup \{\lnot A\} \notin \FINSAT$.

            By definition of finite satisfiability, if $T \cup \{A\} \notin \FINSAT$, then there must exist a \tit{finite} sub-theory $T_0 \subset T \cup \{A\}$ such that $T_0 \in \UNSAT$. Note that $T \in \FINSAT$, therefore if $T \cup \{A\} \notin \FINSAT$ then it must be that $A \in T_0$. Let $\widehat{T_0}$ be the theory such that $T_0 := \widehat{T_0} \cup \{A\}$; then $$T_0 := \widehat{T_0} \cup \{A\} \in \UNSAT \iff \forall \alpha  \quad \alpha(T_0) = 0$$ which implies that $$\forall \alpha \quad \alpha(\widehat{T_0}) = 1 \implies \alpha(A) = 0$$

            Analogously, we can apply the same reasoning for $T \cup \{\lnot A\}$, and we get that there must exist a \tit{finite} sub-theory $T_1 \subset T \cup \{\lnot A\}$ such that $T_1 := \widehat{T_1} \cup \{\lnot A\} \in \UNSAT$, which implies that $$\forall \alpha \quad \alpha (\widehat{T_1}) = 1 \implies \alpha(\lnot A) = 0$$

            Lastly, since $\widehat{T_0} \cup \widehat{T_1} \subset T \in \FINSAT$, by finite satisfiability of $T$ there must exist an assignment $\alpha$ such that $\alpha(\widehat{T_0} \cup \widehat{T_1}) = 1$, and therefore $\alpha(\widehat{T_0}) = \alpha(\widehat{T_1}) = 1$. However, for the previous observations this implies that $\alpha(A) = \alpha(\lnot A) = 0 \ \lightning$.
        }

        Since we are assuming that the propositions of $T$ are \tit{countably infinite}, and the number of variables of each prooposition of $T$ is finite, we can fix an enumeration $p_1, p_2, p_3, \ldots$ on the --- possibly infinite --- propositional variables of $T$. Given this enumeration, define the following \tit{chain} of sub-theories:
        
        \begin{itemize}
            \item $T_0 := T$
            \item $T_{i + 1} := \soe{ll}{T_i \cup \{p_i \} & T_i \cup \{p_i\} \in \FINSAT \\ T_i \cup \{\lnot p_i\} & T_i \cup \{\lnot p_i\} \in \FINSAT}$
        \end{itemize}

        and note that, by definition, clearly $$T =: T_0 \subseteq T_1 \subseteq T_2 \subseteq \ldots$$ Moreover, let $$T^* := \bigcup_{i \in \N}{T_i}$$ and note that since $\forall i \quad T_i \in \FINSAT$ by definition, then it must be that $T^* \in \FINSAT$ as well, as $T^*$ is a chain defined \tit{only} by inclusions of \FINSAT theories.

        Now, consider the following assignment: $$\funcmap{\alpha^*}{\{p_1, p_2, \ldots\}}{\{0,1\}}{p_i}{\soe{ll}{1 & p_i \in T^* \\ 0 & \lnot p_i \in T^*}}$$ Note that this assignment is well defined, because by construction of $T^*$ only one between $p_i \in T^*$ and $\lnot p_i \in T^*$ can hold.

        \claim{
            $\alpha^*(T) = 1$.
        }{
            Let $A \in T$, and let $p_{i_1}, \ldots, p_{i_k}$ the be the propositional variables that appear in $A$. \todo{props finitely many vars?} Then, for each $j \in [k]$ let $$p_{i_j}^* := \soe{ll}{p_{i_j} & p_{i_j} \in T^* \\ \lnot p_{i_j} & \lnot p_{i_j} \in T^*}$$ and consider the set $\cbk{A, p_{i_1}^*, \ldots, p_{i_k}^*}$. Clearly, this is a finite subset of $T^*$ \todo{non sono d'accordo}, hence $T^* \in \FINSAT$ implies that there must exist an assignment $\beta_A$ that satisfies this set, i.e. $$\beta_A(A) = \beta_A \rbk{p_{i_1}^*} = \ldots = \beta_A \rbk{p_{i_k}^*} = 1$$ Note that, for each $j \in [k]$, it holds that $$p_{i_j} \in T^* \implies p_{i_j}^* = p_{i_j} \land \alpha^*\rbk{p_{i_j}} = 1$$ and $1 = \beta_A (p_{i_j}^*) = \beta_A \rbk{p_{i_j}}$; analogously, it holds that $$p_{i_j } \notin T^* \implies p_{i_j}^*= \lnot p_{i_j} \land \alpha^ *\rbk{p_{i_j}} = 0$$ and $1 = \beta_A (p_{i_j}^*) = \beta_A \rbk{\lnot p_{i_j}} = \lnot \beta_A \rbk{p_{i_j}} \implies \beta_A \rbk{p_{i_j}} = 0$. This proves that $\alpha^* \equiv \beta_A$ for all of $A$'s variables, therefore it must also be true that $\alpha^*(A) = \beta_A(A)$. \todo{not enough for whole T? missing last part}
        }

        This claim proves that there exists an assignment $\alpha ^*$ that satisfies $T$, hence $T \in \SAT$, concluding the proof.
    \end{proof}

    The statement of this theorem is equivalent to the following one.

    \begin{framedcor}{}
        Given an infinite theory $T$, and a proposition $A$, it holds that $$T \models A \iff \exists T' \subset T \ \mathrm{finite} \quad T' \models A$$
    \end{framedcor}

    \begin{proof}
        placeholder \todo{prove their equivalence}
    \end{proof}
   
    The compactness theorem can be proven to be equivalent to a special case of \href{https://en.wikipedia.org/wiki/K%C5%91nig%27s_lemma}{Kőnig's lemma} \cite{koenig}, which states the following.

    \begin{framedlem}{Kőnig's lemma (special case)}
        Every infinite tree contains either a vertex of infinite degree, or an infinite path.
    \end{framedlem}

    \printbibliography % UNCOMMENT FOR BIBLIOGRAPHY

\end{document}
